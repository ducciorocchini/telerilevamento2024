\documentclass{beamer}
\usepackage{listings}
% 
\usetheme{Warsaw} % Frankfurt | Warsaw
\usecolortheme{dove} %crane  dove dolphin beaver dove
% % block styles
\setbeamertemplate{blocks}[rounded][shadow=true]
% 
% plot(inputtable)

\title{Choose your preferred title}
 \subtitle{beamer}
 \author{Duccio Rocchini, The PhD students}
 \institute{
 Alma Mater Studiorum UniBO\\
 \bigskip
 \includegraphics[width=0.3\textwidth]{satellite2.jpeg}
 }

\begin{document}
\maketitle

\AtBeginSection[] % Do nothing for \section*
{
\begin{frame}{Outline}
\tableofcontents[currentsection]
\end{frame}
}

\section{Introduction}

\begin{frame}{My very first slide in LaTeX: hoorray!!}
    \texttt{My first text here!!}
\end{frame}

\begin{frame}{My very first slide in LaTeX: hoorray!!}
    \scriptsize{My first text here!!}
\end{frame}

\begin{frame}{My very first slide in LaTeX: hoorray!!}
    \tiny{My first text here!!}
\end{frame}



\begin{frame}{My very first slide in LaTeX: hoorray!!}
    My first text here!! \\
    Here write the second sentence. \\
    Third sentence here.
\end{frame}


\begin{frame}{My very first slide in LaTeX: hoorray!!}
    My first text here!! \\
    \bigskip
    Here write the second sentence. \\
    \bigskip
    Third sentence here.
\end{frame}

\subsection{Itemizing}

\begin{frame}
 \frametitle{Itemizing}
 \begin{itemize}
  \item Remote sensing is a powerful tool
  \item It can be used for diversity estimate
  \item It can be used in ecological informatics
 \end{itemize}
\end{frame}


\begin{frame}
 \frametitle{Itemizing}
 \begin{itemize}
  \item<1> Remote sensing is a powerful tool
  \item<2> It can be used for diversity estimate
  \item<3> It can be used in ecological informatics
 \end{itemize}
\end{frame}

\begin{frame}
 \frametitle{Itemizing}
 \begin{itemize}
  \item<1-> Remote sensing is a powerful tool
  \item<2-> It can be used for diversity estimate
  \item<3-> It can be used in ecological informatics
 \end{itemize}
\end{frame}



\begin{frame}{Itemizing}
\begin{itemize}
    \item first item
    \item second item: I love ecology
    \item I love Biol Geol Env Sci
\end{itemize}
\end{frame}

\begin{frame}{Itemizing}
\begin{itemize}
\tiny{
    \item<1-> first item
    \item<2-> second item: I love ecology
    \item<3-> love Biol Geol Env Sci
}
\end{itemize}
\end{frame}

\section{Figures}
\begin{frame}{Adding a figure}
\centering
 \includegraphics[width=0.9\textwidth]{satellite2.jpeg}
\end{frame}

\begin{frame}{Adding a figure}
\centering
 \includegraphics[width=0.4\textwidth]{satellite2.jpeg}
 \includegraphics[width=0.4\textwidth]{satellite2.jpeg}
\end{frame}

\begin{frame}{Adding a figure}
\centering
 \includegraphics[width=0.4\textwidth]{satellite2.jpeg}
 \includegraphics[width=0.4\textwidth]{satellite2.jpeg}\\
 \includegraphics[width=0.4\textwidth]{satellite2.jpeg}
  \includegraphics[width=0.4\textwidth]{satellite2.jpeg}\\
\end{frame}

\begin{frame}{Adding a figure}
\centering
 \includegraphics[width=0.4\textwidth]{satellite2.jpeg}
 \includegraphics[width=0.4\textwidth]{satellite2.jpeg}\\
 \bigskip
 \includegraphics[width=0.4\textwidth]{satellite2.jpeg}
  \includegraphics[width=0.4\textwidth]{satellite2.jpeg}\\
\end{frame}


\begin{frame}{Adding a figure}
\centering
 \includegraphics[width=0.4\textwidth]{satellite2.jpeg}
 \includegraphics[width=0.4\textwidth]{satellite2.jpeg}\\
 \smallskip
 \includegraphics[width=0.4\textwidth]{satellite2.jpeg}
  \includegraphics[width=0.4\textwidth]{satellite2.jpeg}\\
\end{frame}

\section{Formulas}


\begin{frame}{Formulaaaaaaaas!}
    \begin{equation}
        H' = - \sum_{i=1}^{N} p_i \times \log{p_i}
    \end{equation}
\end{frame}

\begin{frame}{More difficult formulas...}
Let's experience the easiness of writing difficult formulas:
\begin{equation}
K_{\alpha}=\frac{1}{\left( \sum_{i=1}^{N} p_i \times p_i^{\alpha-1} \right)^{\frac{1}{\alpha-1}}} 
\end{equation}
\end{frame}


\begin{frame}{Formulaaaaaaaas!}
Bla bla bla I love formulas, especially into my sentences
$H' = - \sum_{i=1}^{N} p_i \times \log{p_i}$ and blablabla again
\end{frame}

\begin{frame}{More info on formulas...}
\url{https://en.wikibooks.org/wiki/LaTeX/Mathematics}
\end{frame}

\begin{frame}{Formulas that no other software can do!}
    \begin{equation}
    M_d=
    \begin{pmatrix}
    d_{\lambda_1,\lambda_1} & d_{\lambda_1,\lambda_2} & d_{\lambda_1,\lambda_3} & \cdots & d_{\lambda_1,\lambda_n} \\
    d_{\lambda_2,\lambda_1} & d_{\lambda_2,\lambda_2} & d_{\lambda_2,\lambda_3} & \cdots & d_{\lambda_2,\lambda_n} \\
    d_{\lambda_3,\lambda_1} & d_{\lambda_3,\lambda_2} & d_{\lambda_3,\lambda_3} & \cdots & d_{\lambda_3,\lambda_n} \\
    \vdots  & \vdots  & \vdots  & \ddots & \vdots  \\
    d_{\lambda_n,\lambda_1} & d_{\lambda_n,\lambda_2} & d_{\lambda_n,\lambda_3} & \cdots & d_{\lambda_n,\lambda_n}
    \end{pmatrix}
    \end{equation}
\end{frame}

\begin{frame}{The code!}
    \lstinputlisting[language=R]{list2.r}
\end{frame}

\begin{frame}{The code!}
    \lstinputlisting[language=R]{list2.r} \\
    \bigskip
    This is the function of R we are using, where:
    \begin{itemize}
        \item x = input file...
        \item $\alpha$ = blablabla
    \end{itemize}
\end{frame}


 \end{document}
